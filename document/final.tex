\chapter*{Conclusions}
\addcontentsline{toc}{chapter}{Conclusions}

We have introduced the motivations to study noise in genetic circuits, some of the background needed, and some of the models that have been developed together with the mathematical tools used in them. These tools include the Master Equation, the Langevin Equation, and the Fluctuation-Dissipation Theorem. We applied them in a variety of contexts to illustrate their application.

Furthermore, we have shown for some basic circuits such as the simple gene, that the three aproaches lead exactly to the same expression for the noise. But, how do we know in which cases it is better to use one or the other? As we have seen in chapter \ref{ch:master}, to write the ME we need to have specified the possible states for the system and the transitions between them. Then, all the components that affect the noise must be identified. The ME is thus better suited for genetic systems that are characterized with sufficient detail to fulfill these conditions. The FDT is very similar to the ME because both need to be provided with specific information about the circuit.

In chapter \ref{ch:langevin} we used the Langevin approach to include the global noise. It arises as a consequence of many factors that have the common feature of causing correlated changes in every component of the cell. The ME approach does not work in this case because it is very difficult to characterize each of the specific contributions to global noise. However, the Langevin equation allows us to model the noise using more general features about it, e.g. the fact that the global noise is correlated among the different components and intrinsic noise is not. From these apparently simple assumptions very reliable models have been developed that also match the experiments \cite{pedraza05}.

In chapter \ref{ch:fdt} we studied the noise produced by the activation of the genes, in chapter \ref{ch:bursting} the gestation and senescence of the molecules was included in the models, and in chapter \ref{ch:div} the error arising from the partition of molecules at cell division. The wide variety of ways in which noise can be affected has led to the conclusion that by only measuring the noise, it is not possible to infer which mechanisms produce it. Many combinations of different processes can produce the same statistics.

This is also a problem because noise is often measured using fluorescent reporters such as the cyan (\textit{cfp}), yellow (\textit{yfp}), and red (\textit{rfp}) fluorescent proteins. For each cell, the intensity of fluorescence on each color is proportional to the number of the corresponding proteins present. Then, by measuring the individual intensities in a large population and taking the variance and the mean, the noise can be found. However, quantities such as the number of mRNAs can not be directly inferred.

In the first experimental works (including \cite{pedraza05}) the parameters and variables of the models were fitted based on prior assumptions about the sources of the noise. Nevertheless, the experimental observation agree with the analytic equations. On the one hand, it apparently tell us that models with predictive power can be developed without the necessity of having correct prior assumptions. The stochastic modeling of synthetic circuits is favoured by this because although we might be associating the origin of noise to the incorrect sources, they seem to correctly predict the general behavior of randomness. On the other hand, this fact obstructs the understanding of the design principles of genetic circuits, a cornerstone of Systems Biology, by using the traditional techniques to measure noise. If noise measurements does not allow us to characterize their sources, we would not be able to learn about the strategies that evolution has developed to control them.

This has promoted the progress in innovative experimental techniques to count individual molecules in the cells, and to follow their time dynamics with high resolution. For example, in \cite{golding05}, the researchers were able to count mRNA molecules and test some common hypotheses of the models (e.g. Poisson processes, bursts, etc.). Another experimental techniques to count molecules and follow the dynamic of single cells are the the Mother Machine \cite{wang10} and the Microfluidic Assisted Cell Screening (MACS) \cite{okumus13}. There is still a lot of work to do with these techniques, specially on their applications in the testing of the stochastic models.

On the theoretical front, most of the models have been linearized about the steady states. The nonlinearities of biological circuits are very difficult to treat stochastically. However, in these nonlinearities lies part of the rich and complex processes of living beings. For that reason, together with the experimental advancements, additional analytical tools should be developed that allow to consider the nonlinearities and the time dynamics of fluctuations.

Therefore, there are two possible future advancements of this work. First, I would like to learn more about the experimental applications of microfluidic devices to count individual molecules and try to characterize the sources of noise with more detail. Second, I would like to work on the possibility of integration of the models I have studied, and try to develop more general models that include the nonlinearities and/or the time dynamics. I am currently exploring the possibility of using tools from stochastic processes for this purpose.

A specific topic in which I am interested is the study of noise in cell volume and cell division. Most of the processes involved in gene expression depend on the concentration rather than on the number of molecules. If volume were deterministic it is trivial to switch from numbers to concentration. But there is randomness in volume \cite{tanouchi15}, and this has an effect in the concentration of every component of the cell. Also, in the models it is common to approximate the effect of cell division as continous degradation. I am developing simulations that explicitly segregate randomly the components with the purpose of comparing the differences in the noise found with and without this approximation.

Finally, along with the process of learning this subject, I wrote this document with the expectation that the people who wants to learn about noise in gene expression use it as a bibliographical resource. The background knowledge needed to study noise in biology is not extensively taught in the mandatory undergraduate courses, and it does not exist a book about this topic. Therefore, I explained the motivations to study this subject in the introduction, and I devoted an entire chapter to the prelimilary concept. Also, I developed the mathematical derivations with detail, trying to clarify the aspects that were difficult for me to understand when I first read the articles.
