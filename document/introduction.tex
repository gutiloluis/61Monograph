\chapter*{Introduction}
\addcontentsline{toc}{chapter}{Introduction}

Genetic information determines the constitution and function of the cells. However, genetically identical cells show differences, such as cells from our skin and our neurons. This is explained by the mechanisms of gene regulation that determine the levels of expression of the different genes, that may change e.g. according to environmental signals even for identical cells. The chemical reactions that regulate gene expression are, at a fundamental level, collisions between diffusing molecules that occur randomly. Furthermore, some of the molecules involved (such as mRNA molecules), are present at low numbers increasing the randomness.

It is surprising how all these random chemical reactions are able to determine such complex and synchronized patterns in living beings. There must be sophisticated mechanisms in the genetic circuits that allow them to work properly regardless of the presence of noise. It is even more surprising that living beings have taken advantage of noise to introduce variability in 


Stochasticity, or noise, in biological circuits occurs due to of fluctuations during transcription, translation \cite{kaern05} and other processes that affect gene expression. As a consequence of noise, genetically identical cells and on the same environment may have notorious phenotypical variations \cite{kaern05} \cite{elowitz02} \cite{pedraza05}. This noise has been classified in two groups: intrinsic and extrinsic \cite{elowitz02} \cite{paulsson05}. The former is the variability inherent to systems with discrete components and low numbers (e.g. RNA and proteins). The latter is related to external factors as environmental fluctuations, cell growing and cell division.

Recent works have shown the importance of noise for living beings. They have adapted their genetic circuits to develop their respective functions correctly regardless of its presence (robustness) \cite{alon99}, or to take advantage of it to produce variability \cite{arkin98}. Also, when designing synthetic genetic circuits it is important to consider the stochasticity that the circuit may have.

For those reasons, in the last years, several stochastic models of gene expression have been developed. In a pioneer work, Thattai and van Oudenaarden \cite{thattai01} a linearized model for intrinsic noise in the amounts of RNA and proteins that can be applied to some basic motifs. Also, Pedraza and van Oudenaarden \cite{pedraza05} developed a model that includes extrinsic noise and showed how fluctuations are propagated through a cascade of regulation.

Most recent models have focused in other aspects that could induce noise. For instance, the bursting in the production of the molecules involved in gene expression, their senescence \cite{pedraza08}, and the partition of molecules during cell division \cite{huh11a} \cite{huh11b}. One of the most important conclusions of these works is that when considering different factors, the behavior of noise is similar. Therefore, by studying only the fluctuations it is difficult to know the mechanisms that produce them.

Altought many important results have been made, most of the models used are linearized around the fixed points due to the non-linearity of the equations used to model molecular kinetics. With this, information about the full dynamics of fluctuations. it would be useful then to develop stochastic models that consider the non-linearities, that include the time evolution of noise and that consider more factors like the cell growing and division together with gene expression.
