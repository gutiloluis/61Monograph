\chapter*{Introduction}
\addcontentsline{toc}{chapter}{Introduction}

La estocasticidad o ruido en circuitos gen\'eticos se da debido a las fluctuaciones durante la transcripci\'on, traducci\'on \cite{kaern05} y otros procesos que afectan la expresi\'on gen\'etica. Debido a ella, c\'elulas gen\'eticamente id\'enticas y expuestas al mismo ambiente presentan variaciones fenot\'ipicas notables \cite{kaern05} \cite{elowitz02} \cite{pedraza05}. Este ruido ha sido clasificado en dos grupos: intr\'inseco y extr\'inseco \cite{elowitz02} \cite{paulsson05}. El primero hace referencia a la variabilidad inherente a sistemas con componentes discretos y en bajos n\'umeros, como lo son en este caso el ARN y las prote\'inas. El segundo hace referencia a factores externos como la variabilidad del ambiente, el crecimiento y la divisi\'on celular.

Recientes investigaciones han mostrado la enorme importancia que tiene el ruido para los seres vivos, los cuales han adaptado sus circuitos gen\'eticos ya sea para cumplir su funci\'on adecuadamente a pesar de la presencia de ruido (robustez) \cite{alon99}, o para aprovecharlo para generar variabilidad \cite{arkin98}. De igual manera, en el desarrollo de circuitos gen\'eticos sint\'eticos es importante considerar la estocasticidad que dicho circuito podr\'ia presentar.

Lo anterior ha motivado al desarrollo de modelos estoc\'asticos de expresi\'on gen\'etica en los \'utimos a\~nos. En el trabajo pionero de Thattai y van Oudenaarden,  \cite{thattai01} se realiz\'o un modelo linealizado para el ruido intr\'inseco en la cantidad de ARN y prote\'inas que puede aplicarse a varios circuitos b\'asicos. Tambi\'en, Pedraza y van Oudenaarden \cite{pedraza05} desarrollaron un  modelo que incluye el ruido global y mostraron c\'omo el ruido total se propaga a trav\'es de una cascada de regulaci\'on.

Los modelos m\'as recientes se han centrado en considerar otros aspectos que podr\'ian modificar las caracter\'isticas del ruido. Entre ellos se encuentran el \textit{bursting} en la producci\'on de las mol\'eculas involucradas en la expresi\'on, su senescencia \cite{pedraza08} y su repartici\'on durante la divisi\'on celular \cite{huh11a} \cite{huh11b}. Una de las conclusiones m\'as importantes que se han obtenido a partir de estos modelos es que al considerar distintos factores, el comportamiento del ruido es similar y por lo tanto lo son tambi\'en los ajustes de los modelos num\'ericos y los experimentos. Por lo tanto a partir de las caracter\'isticas de las fluctuaciones no se puede saber precisamente cu\'ales son los mecanismos que las producen.

A pesar de que se han obtenido resultados importantes, la alta no linealidad de las ecuaciones utilizadas para representar la cin\'etica molecular ha obligado a que los modelos realizados sean en su mayor\'ia linealizados alrededor de los puntos fijos, perdiendo as\'i informaci\'on acerca de la din\'amica completa de las fluctuaciones. Ser\'ia \'util entonces desarrollar modelos estoc\'asticos que no ignoren las no-linealidades, que incluyan la din\'amica temporal completa y que integren m\'as factores como el crecimiento y la divisi\'on celular.

\section{Objetivo general}

Estudiar detalladamente los principales modelos estoc\'asticos de expresi\'on gen\'etica.
