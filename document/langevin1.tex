\chapter{Cascade of regulation - The Langevin equation}

We will consider a set of genes whose interactions are shown on figure FILL. We will consider both intrinsic and extrinsic sources of noise. The intrinsic part refers to the inherent noise due to the low number of molecules and the nature of the reactions. This was the only source of noise consider on the previous chapter. The extrinsic part arises from another factors, such as environmental changes or sudden changes in intracellular concentrations. The fluctuations due to those sources of noise in the different genes are correlated, while the fluctuations due to intrinsic noise are not (EXPLAIN BETTER, PERHAPS ADD A SECTION ON PRELIMINARY CONCEPTS).

The differential equation for the mRNA will not be considered, we will write the equation for the proteins and include the effect of the mRNA in the rate of creation $k$ As we have seen, the deterministic equation for the number of proteins of gene $0$ is 

\begin{equation}
\label{eq:detgene0}
\dot{x_0}(t) = k - \gamma x_0(t).
\end{equation}

Where now $k$ represents the average number of proteins created per unit time, and not the proteins per mRNA per unit time as on the previous chapter. In this approach we add noise terms to the previous equation, one representing the intrinsic noise $\mu_0(t)$ and other representing the global noise $\xi_0(t)$. The stochastic equation then becomes

\begin{equation}
\label{eq:gene0}
\dot{x_0}(t) = k - \gamma x_0(t) + \mu_0(t) + \xi_0(t).
\end{equation}

Now the quantities are taken to be stochastic processes. To find the correlations and the coefficient of variation, we need some information about the noise terms. First, the average of proteins $\langle x_0 \rangle (t)$ should follow the deterministic equation \ref{eq:detgene0}, therefore

\begin{equation}
\langle\mu_0\rangle = \langle\xi_0\rangle = 0.
\end{equation}

Second, we will assume that both sources are white noise, that is, the values of the noise terms at different times are uncorrelated, (EXPLAIN MORE ABOUT THE CONSTANTS AND THEIR DEFINITIONS), that is written as

\begin{align}
\langle\mu_0(t)\mu_0(t+\tau)\rangle&=2\gamma(b_0+1)\bar{x_0}\delta(\tau),\label{eq:corin0}\\
\langle\xi_0(t)\xi_0(t+\tau)\rangle&=2\gamma\eta_G^2\bar{x_0}^2\delta(\tau). \label{eq:corex0}
\end{align}

where $\eta_G$ is the strenght of the global noise, a parameter that is measured experimentally, and $b$ is the average number of protein produced per mRNA. In this section the bar will denote steady state average. Also, since both sources of noise are uncorrelated

\begin{equation}
\label{eq:corinex0}
\langle\mu_0(t)\xi_0(t+\tau)\rangle = 0.
\end{equation}

Proceeding with the calculation of the coefficient of variation, define $\delta x_0 \coloneqq x_0 - \bar{x_0}$, replacing this on eq. \ref{eq:gene0} we get

\begin{equation}
\frac{d}{dt}\left(\delta x_0(t) + \bar{x_0}\right) = k - \gamma (\delta x_0(t) + \bar{x_0}) + \mu_0(t) + \xi_0(t),
\end{equation}

Using the fact that $x_0=\nicefrac{k}{\gamma}$ we get

\begin{equation}
\label{eq:dgene0}
\dot{\delta x_0}(t) = -\gamma \delta x_0(t) + \mu_0(t) + \xi_0(t).
\end{equation}

We will Fourier transform the equation, solve for $\delta x_0$, find its square norm and use the Wiener-Khinchin theorem to find the autocorrelations in terms of the power spectrum and viceversa (EXPLAIN MORE - SEE PREVIOUS CHAPTER, ADD WK TH. ON PREL. ADD FOURIER ?).

Taking the Fourier transform of eq. \ref{eq:dgene0} and recalling that $[\mathscr{F}(\dot{x(t)})](\omega) = i\omega \mathscr{F}(x(t))(\omega)$ for a function of time $x$, we obtain after solving for $\hat{\delta x_0}$

\begin{equation}
\label{eq:fgene0}
\hat{\delta x_0}(\omega) = \frac{\hat{\mu_0}+\hat{\xi_0}}{\gamma + i\omega}.
\end{equation}

Taking the square norm and averaging we get

\begin{equation}
\left\langle |\hat{\delta x_0}|^2 \right\rangle = \frac{\left\langle|\hat{\mu_0}|^2\right\rangle + \left\langle\hat{\mu_0}^*\hat{\xi_0}\right\rangle+\left\langle\hat{\mu_0}\hat{\xi_0}^*\right\rangle+\left\langle|\hat{\xi_0}|^2\right\rangle}{\gamma^2 + \omega^2}
\end{equation}

And using the Wiener-Khinchin theorem and eqs. \ref{eq:corin0} - \ref{eq:corinex0} (EXPLAIN MORE?)

\begin{equation}
\label{eq:pgene0}
\begin{split}
\left\langle |\hat{\delta x_0}|^2 \right\rangle &= \frac{\left(2\gamma(b_0+1)\bar{x_0}+ 2\gamma\eta_G^2\bar{x_0}^2\right)\mathscr{F}(\delta(t))}{\gamma^2+\omega^2}\\
&=\frac{2\gamma\bar{x_0}^2\left(\nicefrac{(b_0+1)}{\bar{x_0}}+ \eta_G^2\right)}{\gamma^2+\omega^2},
\end{split}
\end{equation}

where the cross terms are zero by eq. \ref{eq:corinex0}. Now making the inverse Fourier transform at $t=0$ we obtain the variance.

\begin{equation*}
\langle \delta x_0^2 \rangle = 2\gamma\bar{x_0}^2\left(\nicefrac{(b_0+1)}{\bar{x_0}}+ \eta_G^2\right)\frac{1}{2\pi}\int_{-\infty}^{\infty}\frac{d\omega}{\omega^2+\gamma^2}
\end{equation*}

The integral can be easily solved by residues and the result is $\nicefrac{\pi}{\gamma}$, therefore

\begin{equation*}
\langle \delta x_0^2 \rangle = \bar{x_0}^2\left(\frac{(b_0+1)}{\bar{x_0}}+ \eta_G^2\right)
\end{equation*}

And dividing by $\bar{x_0}^2$, we obtain the coefficient of variation

\begin{equation}
  \label{eq:etagene0}
  \boxed{\eta_0^2 = \frac{(b_0+1)}{\bar{x_0}}+ \eta_G^2 = \eta_{0\text{int}}^2+\eta_G^2}.
\end{equation}

This approach enabled us to explicitly separate the total noise of gene $0$ in the intrinsic and the extrinsic part. Now we will make the calculation for gene $1$, which follows the equation.

\begin{equation}
\label{eq:gene1}
\dot{x_1}(t) = k_1(x_{0A})-\gamma x_1+\mu_1+\xi_1
\end{equation}

The decay rate $\gamma$ is the same for gene $1$ than for gene $0$ after making the assumption that the decay is ruled by dilution due to cellular growth. The creation rate $k_1$ is a Hill equation for activation. The statistics for the noise terms are analogous to eqs. \ref{eq:corin0} - \ref{eq:corinex0}. We also need to know in this case the correlations between the noise terms corresponding to gene $0$ and the ones corresponding to gene $1$. As we have said, extrinsic sources of noise are uncorrelated

\begin{equation}
\label{eq:corcross01}
\langle\mu_0(t)mu_1(t+\tau)\rangle = \langle\mu_0(t)\xi_1(t+\tau)\rangle = \langle\mu_1(t)\xi_0(t+\tau)\rangle = 0,
\end{equation}

but the extrinsic parts of the noise of genes $0$ and $1$ are correlated. In analogy with eq. \ref{eq:corex0} (EXPLAIN) we get

\begin{equation}
  \langle\xi_0(t)\xi_1(t+\tau)\rangle = 2\gamma\eta_G^2\bar{x_0}\bar{x_1}\delta(\tau).
\end{equation}

Now we will proceed in a similar way to gene $0$. Defining $\delta x_1(t) \coloneqq x_1(t) - \bar{x_1}$ and writing eq. \ref{eq:gene1} in terms of $\delta x_1$, $\delta x_{0A}$, and making a Taylor expansion of $f_1$ to first order in $x_{0A}$ we obtain.

\begin{equation}
\dot{\delta x_1} = k_1(\bar{x_{0A}}) + \left.\frac{dk_1(x_{0A})}{dx_{0A}}\right|_{\bar{x_{0A}}}\delta x_{0A} - \gamma(\delta x_1 + \bar{x_1}) + \mu_1 + \xi_1,
\end{equation}

but from eq. \ref{eq:gene1} we can see that $\bar{x_1} = \nicefrac{k_1(\bar{x_{0A}})}{\gamma}$, therefore

\begin{equation}
  \label{eq:dgene1}
  \dot{\delta{x_1}(t)}=c_1\delta x_{0A}-\gamma\delta x_1 + \mu_1 \xi_1,
\end{equation}

where $c_1 \coloneqq \left.\nicefrac{dk_1(x_{0A})}{dx_{0A}}\right|_{\bar{x_{0A}}}$ Fourier transforming and solving for $\hat{\delta x_1}$ we get

\begin{equation*}
  \hat{\delta x_1}=\frac{c_1\hat{\delta x_{0A}}+\hat{\mu_1}+\hat{\xi_1}}{\gamma + i\omega}.
\end{equation*}

Taking the square norm and averaging

\begin{equation}
  \label{eq:pgene1}
  \begin{split}
    \left\langle|\hat{\delta x_1}|^2\right\rangle &= \frac{1}{\omega^2+\gamma^2}\left(c_1\hat{\delta x_{0A}} + \hat{\mu_1} + \hat{\xi_1}\right)\left(c_1\hat{\delta x_{0A}}^* + \hat{\mu_1}^* + \hat{\xi_1}^*\right)\\
    &=\frac{1}{\omega^2+\gamma^2}\left(c_1^2 \left\langle|\hat{\delta x_{0A}}|^2\right\rangle + c_1\left(\langle\hat{\delta x_{0A}}\hat{\xi_1}^*\rangle+\langle\hat{\delta x_{0A}}^*\hat{\xi_1}\rangle\right) +  \left\langle|\hat{\mu_1}|^2\right\rangle +  \left\langle|\hat{\xi_1}|^2\right\rangle\right)
  \end{split}
\end{equation}

Using the Wiener-Khinchine theorem and the equations for the correlations we get

\begin{align}
  \left\langle|\hat{\mu_1}|^2\right\rangle &= 2\gamma(b_1+1)\bar{x_1},\\
  \left\langle|\hat{\xi_1}|^2\right\rangle &= 2\gamma\eta_G^2\bar{x_1}^2,\\
\end{align}

since the Fourier transform of the Dirac delta is $1$. Also, from eqs. \ref{eq:fgene0} and \ref{eq:pgene0} we get

\begin{align}
\left\langle|\hat{\delta x_{0A}}|^2\right\rangle &= \frac{2\gamma\bar{x_0}^2\left(\nicefrac{(b_0+1)}{\bar{x_0}}+ \eta_G^2\right)}{\gamma^2+\omega^2},\\
\langle\hat{\delta x_{0A}}\hat{\xi_1}^*\rangle &= \frac{1}{\gamma+i\omega}\left(\langle\hat{\mu_0}\hat{\xi_1}^*\rangle + \langle\hat{\xi_0}\hat{\xi_1}^*\rangle \right) = \frac{\langle\hat{\xi_0}\hat{\xi_1}^*\rangle}{\gamma+i\omega}\\
\langle\hat{\delta x_{0A}}^*\hat{\xi_1}\rangle &= \frac{1}{\gamma-i\omega}\left(\langle\hat{\mu_0}^*\hat{\xi_1}\rangle + \langle\hat{\xi_0}^*\hat{\xi_1}^*\rangle \right) = \frac{\langle\hat{\xi_0}^*\hat{\xi_1}\rangle}{\gamma-i\omega}
\end{align}

Where the last step in the last two equations comes from the Wiener-Khinchin theorem and eq. \ref{eq:corcross01}. Replacing the previous equations in eq. \ref{eq:pgene1} and taking the inverse transform we get for the variance

\begin{equation}
  \begin{split}
    \langle \delta x_1^2\rangle &= 2\gamma\bar{x_0}^2c_1^2\left(\nicefrac{(b_0+1)}{\bar{x_0}}+ \eta_G^2\right)\frac{1}{2\pi}\int_{-\infty}^{\infty}\frac{d\omega}{(\omega^2+\gamma^2)^2}\\
    &+2\gamma\eta_G^2\bar{x_0}\bar{x_1}c_1\frac{1}{2\pi}\left(\int_{-\infty}^{\infty}\frac{d\omega}{(\gamma + i\omega)(\omega^2+\gamma^2)} + \int_{-\infty}^{\infty}\frac{d\omega}{(\gamma - i\omega)(\omega^2+\gamma^2)}\right)\\
    &+2\gamma\bar{x_1}^2 \left(\nicefrac{(b_1+1)}{\bar{x_1}}+\eta_G^2\right)\frac{1}{2\pi}\int_{-\infty}^{\infty}\frac{d\omega}{\omega^2+\gamma^2}.
  \end{split}
\end{equation} 

Solving the integrals in the complex plane and rearranging we get.

\begin{equation}
  \langle \delta x_1^2\rangle = \frac{c_1^2\bar{x_0}^2}{2\gamma^2}\left(\nicefrac{(b_0+1)}{\bar{x_0}}+ \eta_G^2\right) + \frac{c_1\eta_G^2\bar{x_0}\bar{x_1}}{\gamma} + \bar{x_1}^2\left(\nicefrac{(b_1+1)}{\bar{x_1}}+\eta_G^2\right).
\end{equation}

Writing in terms of the logarithmic gain (EXPLAIN!!, ASK), $H_{10}=-\frac{c_1\bar{x_0}}{\gamma x_1}$, dividing by $\bar{x_1}^2$ and rearranging we get

\begin{equation}
  \label{eq:etagene1}
  \boxed{\eta_1^2 = \eta_{1\text{int}}^2 + \frac{1}{2}H_{10}^2\eta_0^2+\eta_G^2\left(1-H_{10}\right)}.
\end{equation}

Where $\eta_{1\text{int}}^2 = \nicefrac{(b_1+1)}{\bar{x_1}}$ and $\eta_0$ is given by eq. \ref{eq:etagene0}.

The result can be interpreted as follows, the total noise in gene one is givin by the intrinsic part, the noise from gene $0$ that is propagated to gene $1$ (including its intrinsic and global part) and the global noise that enters directly into gene $1$. The factor of $\nicefrac{1}{2}$ arises from the time averaging (EXPLAIN). MORE ANALYSIS, GRAPHICS, ETC.

For gene two we proceed similarly, with analogous statistics for the noise terms, the resulting noise is (CITE PAPER)

\begin{equation}
  \label{eq:etagene2}
  \boxed{\eta_2^2 = \eta_{2\text{int}}^2 + \frac{1}{2}H_{21}^2\eta_1^2+\frac{1}{8}H_{21}^2H_{10}^2\eta_0^2+\eta_G^2\left(1-H_{21}-\frac{1}{4}H_{21}^2H_{10}+\frac{1}{2}H_{21}H_{10}\right)}.
\end{equation}

Which contains the intrinsic noise of gene $2$, the contribution from the total noise of gene $1$, the contribution from the total noise of gene $0$ that is transmitted first to gene $1$ and then to gene $2$ and the global noise that enters directly.

The correlations can be found in a similar fashion (DO THAT?)

This approach enables us to calculate the coefficient of variation for a cascade of regulation and separate the different sources of noise. Also, it enables to write the effect of the upstream genes in terms of the logarithmic gain, making it very intuitive. The results of the theoretical model were tested with experiments where the genes that are part of the cascade are transcribed bicistronicaly with fluorescent reporters. The fluctuations in the intensity of the reporters was used to measure the noise in the population of cells.
