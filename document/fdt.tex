\chapter{The Fluctuation-Dissipation theorem (FDT)}

This chapter is based on the work of J. Paulsson in \cite{paulsson04} and \cite{paulsson05}

\todo[inline]{Do this chapter}

\section{Statement of the FDT}

\todo[inline]{Prove it}

Consider a (genetic) system with $N$ species whose concentrations are $n_1,\dotsc,n_N$. The FDT states that if $\mathbf{\sigma}$ is the matrix of covariances (i.e. $\sigma_{ij} \coloneqq \langle n_i-\langle n_i\rangle\rangle\langle n_j-\langle n_j\rangle\rangle$), then it follows that 

\begin{equation}
  \label{eq:fdt-fdt1}
  \frac{\mathrm{d}\mathbf{\sigma}}{\mathrm{d}t} = \mathbf{A\sigma} + \mathbf{\sigma A^T}+\mathbf{B}.
\end{equation}

Where,

\begin{equation*}
  A_{ij} \coloneqq \frac{\partial}{\partial \langle n_j}\frac{\partial \langle n_i\rangle}{\partial t}
\end{equation*}

and

\begin{equation*}
  B_{ij} = \sum_k v_{jk}v_{ik}R_k,
\end{equation*}

where $k$ runs over all the possible reactions for the system. $R_k$ is the rate of reaction $k$, which produces $v_{ik}$ molecules of species $i$.

In steady state, the FDT becomes

\begin{equation}
  \label{eq:fdt-fdtss}
  \mathbf{A\sigma}+\mathbf{\sigma A^T}+\mathbf{B} = 0
\end{equation}

where $\mathbf{A}$ and $\mathbf{B}$ are now evaluated at steady state. We will illustrate the concept with an example in the next section.

\section{Example: simple gene}

Consider a single gene system such as the one of section \ref{sec:mas-single_gene}. We will treat the same case without the suposition of a fixed number of active copies of the gene. In this case, there is a fixed number $n_1^{\text{max}}$ and $n_1$ copies that are active, that is, they are available for transcription. There are several ways in which genes can be turned on and off including binding of transcription factors or chromatin remodeling.

\todo[inline]{Explain chromatin remodeling}

We will approximate the activation of genes in the following scheme, each gene can switch independenty from \textit{off} to \textit{on} with rate $k_1^+$ and in the opposite sense with rate $k_1^-$.

\todo[inline]{Explain telegraph process}

The number of active genes $n_1$ follows a binomial distribution with the probability of being on given by $P_{\text{on}}=k_1^+/(k_1^++k_1^-)$.

For the number of mRNA $n_2$ and proteins $n_3$, we will use the same scheme as on section \ref{sec:mas-single_gene}. Recalling that the averages satisfy the deterministic equations, they follow

\todo[inline]{Explain the eq for n1, define tau1}

\begin{equation}
  \begin{split}
    \dot{\langle n_1\rangle} &= k_1^+n_1^\text{max} - \frac{1}{\tau_1}\langle n_1\rangle\\
    \dot{\langle n_2\rangle} &= k_2^+\langle n_1\rangle - \frac{1}{\tau_2}\langle n_2\rangle\\
    \dot{\langle n_3\rangle} &= k_3^+\langle n_2\rangle - \frac{1}{\tau_3}\langle n_2\rangle
  \end{split}
\end{equation}

Here $k_2$ and $k_3$ are the rates of transcription per active gene and translation per mRNA, respectively. Here we write the degradation in terms of $\tau_i\coloneqq 1/\gamma_i$, for $i=2,3$ with the purpose of making the subsequent interpretations clearer. A master equation can be easily written and treated using the methods presented on chapter \ref{ch:master}. Here we will use the FDT in order illustrate it.

First we need to write the matrices $\mathbf{A}$ and $\mathbf{B}$, according to their definition, we have for instance

\begin{equation*}
  \begin{split}
    A_{11} &= \partial_{n_1}\dot{\langle n_1\rangle} = -\frac{1}{\tau_1},\\
    A_{12} &= \partial_{n_1}\dot{\langle n_2\rangle} = k_2^+
  \end{split}
\end{equation*}

and so on, notice that there are several terms that are zero, including $A_{12}$, $A_{13}$, $A_{21}$, etc. Evaluating all the elements results in

\begin{equation}
  \label{eq:fdt-A}
  \mathbf{A} = \begin{pmatrix}
    -1/\tau_1 & 0 & 0 \\
    k_2 & -1/\tau_2 & 0 \\
    0 & k_3 & -1/\tau_3
  \end{pmatrix}.\\
\end{equation}

For the matrix $\mathbf{B}$, there are six reactions in this case: activation and deactivation of a gene, synthesis and destruction of mRNA, and the same for proteins. There are not reactions involving different species neither creating or destroying more than one molecule, thus

\begin{equation*}
  v_{ik}v_{jk} = 
  \begin{cases}
    1 \text{ for } i = j,\\
    0 \text{ for } i\neq j.
  \end{cases}
\end{equation*}

Where the first case stands because  $v_{ik} =\pm 1$ depending on whether the reaction is of synthesis or destruction, but in either case $v_{ik}^2 = 1$.

The rates $R_k$ are given by the deterministic equations. In steady state the rates of creation and destruction are equal. For example, for the reactions involving modifications in $n_2$:

\begin{equation*}
  B_{22} \coloneqq \sum_kv_{2k}^2R_k = \sum_kR_k = k_2\langle n_1\rangle + \frac{1}{\tau_2}\langle n_2\rangle = \frac{2\langle n_2\rangle}{\tau_2},
\end{equation*}

where the last equality holds for the steady state assumption. For $B_{33}$ it is analogous. For $B_{11}$,

\begin{equation*}
  B_{11} = k_1^+(n_1^\text{max}-\langle n_1\rangle)+k_1^-\langle n_1\rangle = 2k_1^-\langle n_1\rangle = 2\frac{1-P_\text{on}}{\tau_1}\langle n_1\rangle,
\end{equation*}

since

\begin{equation*}
  k_1^- = \left(1-\frac{k_1^+}{k_1^++k_1^-}\right)(k_1^++k_1^-)=\frac{1-P_\text{on}}{\tau_1}.
\end{equation*}

Putting all the expressions together, $\mathbf{B}$ becomes

\begin{equation}
  \label{eq:fdt-B}
  \mathbf{B} = 
  \begin{pmatrix}
    2(1-P_\text{on})\langle n_1\rangle/\tau_1 & 0 & 0 \\
    0 & 2\langle n_2\rangle/\tau_2 & 0 \\
    0 & 0 & 2\langle n_3\rangle/\tau_3
  \end{pmatrix}.\\
\end{equation}

Replacing on \eqref{eq:fdt-fdtss} we obtain a linear system that can be solved for $\mathbf{\sigma}$ using a computer program. The diagonal elements are the variances, from which the noises can be found. The next expressions for the coefficients of variation can be trivially found following the procedure, although the calculation may be tedious. The results are

\begin{align}
  \eta_1^2 &= \frac{1-P_\text{on}}{\langle n_1\rangle}, \label{eq:fdt-eta1}\\
  \eta_2^2 &= \frac{1}{\langle n_2\rangle}+\frac{1-P_\text{on}}{\langle n_1\rangle}\frac{\tau_1}{\tau_1+\tau_2}, \label{eq:fdt-eta2}\\
  \eta_3^2 &= \frac{1}{\langle n_3\rangle} + \frac{1}{\langle n_2\rangle}\frac{\tau_2}{\tau_2+\tau_3}+\frac{1-P_\text{on}}{\langle n_1\rangle}\frac{\tau_2}{\tau_2+\tau_3}\frac{\tau_1}{\tau_1+\tau_3}\frac{\tau_1+\tau_3+\tau_1\tau_3/\tau_2}{\tau_1+\tau_2}. \label{eq:fdt-eta3}.
\end{align}

For eq. \eqref{eq:fdt-eta1} the relative width of the distribution is smaller that Poissonian for a given value of $\langle n_1\rangle$. If $P_\text{on}$ is close to $1$ the noise could be low even if $n_1$ is small. This is characteristic of a binomial process.

The first term in eq. \eqref{eq:fdt-eta2} is the Poissonian noise (see eq. \eqref{eq:con-poisson_noise}). The second term is the noise arising from gene activation multiplied by a factor of time averaging. That depends on the decay rates of

\todo[inline]{Time average}

\section{The logarithmic gain}

A quantity that will be very important for the analysis of genetic circuits is the logarithmic gain or elasticity $H_{ij}$. 
\label{sec:log_gain}
