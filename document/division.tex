\chapter{Effects of cell division}
\label{ch:div}

When the cell divides, all the components (organelles, proteins, genetic material, etc.) must be distributed among the daughter cells. Nevertheless, the assymetries on partition are an important source of noise even for components present at high numbers. In this chapter we will explore some general mechanisms of partition of molecules during cell division and how they can affect noise statistics.

This chapter follows the work done by D. Huh and J. Paulsson in \cite{huh11b}.

\section{Characterizing the noise arising from cell division}

Let $x = l+r$ be the number of copies of some component (e.g. a certain protein) for a dividing cell, with $l$ and $r$ being the number of copies each daughter cell recieves. Also, let $v$ be the number of molecules of some component that affects the partition such as vacuoles or spindles. On average, we expect the molecules to distribute symmetrically. Therefore

\begin{equation}
  \label{eq:div-even}
  \langle l\rangle = \langle r\rangle = \frac{\langle x\rangle}{2}.
\end{equation}

We will find the noise for $l$ \footnote{It does not matter if we choose $r$ or $l$ since there is not preference for one of the daughter cells}. Using the law of total variance (eq. \eqref{eq:con-total_var}), the variance of $l$ is given by

\begin{equation*}
  \sigma^2(l) = \sigma^2\left(\langle l|x,v\rangle\right) + \left\langle\sigma^2(l|x,v)\right\rangle.
\end{equation*}

From eq. \eqref{eq:div-even}, $\langle l|x,v\rangle = \frac{x}{2}$, using this and dividing by $\langle l\rangle^2 = \frac{\langle x\rangle^2}{4}$ we get

\begin{equation}
  \label{eq:div-noises}
  \begin{split}
    \eta^2(l) &= 4\frac{\sigma^2\left(\nicefrac{x}{2}\right)}{\langle x\rangle^2}+\frac{\langle\sigma^2(L|x,v)\rangle}{\langle L\rangle^2}\\
    &=\eta^2(x) + Q_x^2,
  \end{split}
\end{equation}

where we used eq. \eqref{eq:con-var_const}. The term $Q_x$, defined as

\begin{equation}
  \label{eq:div-Q}
  Q_x^2 \coloneqq \frac{\langle\sigma^2(L|x,v)\rangle}{\langle L\rangle^2},
\end{equation}

is the noise arising from cell division. Eq. \eqref{eq:div-noises} states that the noise after cell division is the sum of the noise before division and the noise arising at the division process (all squared).

The term $Q_x$ can be interpreted in another way. From the definition of variance and eq. \eqref{eq:div-even}

\begin{equation*}
  Q_x^2 = \frac{1}{\langle l\rangle^2}\left\langle\left\langle\left(l-\langle l\rangle\right)^2|x,v\right\rangle\right\rangle = \frac{4}{\langle x\rangle^2}\left\langle\left\langle\left.\left(l-\frac{x}{2}\right)^2\right|x,v\right\rangle\right\rangle
\end{equation*}

but $l - \frac{x}{2} = \frac{1}{2}(2l-(l-r)) = \frac{l-r}{2}$, then

\begin{equation*}
   Q_x^2 = \frac{4}{\langle x\rangle^2}\left\langle\left\langle\left.\left(\frac{l-r}{2}\right)^2\right|x,v\right\rangle\right\rangle  = \frac{1}{\langle x\rangle^2}\left\langle\left(l-r\right)^2\right\rangle.
\end{equation*}

Therefore, the term $Q_x^2$ can be interpreted as the average square deviation between the quantities of the molecules of each daughter cell. For a perfect division $l=r$, making $Q_x=0$. On the contrary, the most noisy case is when one daughter recieves $x$ moleucles and the other does not recieve any molecule.

In the subsequent derivations, the diverse mechanisms related to cell division will be considered. To simplify the derivations, the particular assumptions used may not match the exact dynamics of the process, but they yield the same noise statistics. For instance, when molecules are grouped in vesicles before being segregated to each half, they are first segregated randomly into vesicles and then vesicles are distributed between the daughters. We will see that for the derivations it will be better to invert the order of the process. Altough this does not correspond to the reality, the same noise is obtained more easily. The authors called those specific models mock processes.

\section{Independent segregation}

In the case of independent segregation, each molecule has an equal probability per unit time to switch from a cell half to another. Assuming there are $l$ and $x-l$ molecules in each half, a process that can describe this statistic is given by

\begin{equation}
  \label{eq:div-arr_ind}
  \begin{split}
    l&\xrightarrow{x-l}l+1\\
    l&\xrightarrow{l}l-1
  \end{split}
\end{equation}

Using the FDT, the jacobian and diffusion $1\times1$ matrices are given by

\begin{equation}
  \begin{split}
    \mathbf{A} &= \partial_l\left((x-l)-l\right) = -2,\\
    \mathbf{B} &= (x-l)+l = x.
  \end{split}
\end{equation}

Solving for the covariance matrix, which in this case is the variance, we obtain in steady state

\begin{equation}
  \label{eq:div-var_ind}
  \sigma^2(l|x) = \frac{x}{4},
\end{equation}

and averaging and using eq. \eqref{eq:div-Q} we get

\begin{equation}
  Q_x^2 = \frac{4}{\langle x\rangle^2}\left\langle\sigma^2(l|x)\right\rangle = \frac{4}{\langle x\rangle^2}\frac{\langle x\rangle}{4} = \frac{1}{\langle x\rangle}.
\end{equation}

In the following sections, we will find the noise for some common division mechanisms and compare it to the case of independent segregation. The mechanisms that increase the noise with respect to it are included in what we call ``disordered segregation'', and we call those that decrease it ``ordered segregation''.

\section{Disordered segregation}

\subsection{General case}

First we will consider a general case in which the rate with which each molecules goes from a cell half to the other is proportional to the available space generated by the upstream component. For a fixed number $v$ of molecules of the upstream component, there are $n$ and $v-n$ available spaces in each daughter cell independently of $x$. Therefore, the process can be written as

\begin{equation}
  \label{eq:div-arr_disg}
  \begin{split}
    l&\xrightarrow{n(x-l)}l+1\\
    l&\xrightarrow{(v-n)l}l-1
  \end{split}
\end{equation}

From the law of total variance we have

\begin{equation}
  \label{eq:div-varlgxv}
  \sigma^2(l|x,v) = \left\langle\sigma^2(l|x,v,n)\right\rangle_{(n|v)}+\sigma^2\left(\langle l|x,v,n\rangle\right)_{(n|v)}
\end{equation}

Where the subscript $(n|v)$ denotes that averages and variances are evaluated over the conditional PDF of $n$ given $x$. Notice that taking averages over $(n|v)$ and over $(n|x,v)$ is the same in this case by the assumption that $n$ is independent of $x$. Also, by symmetry we can assume that $\langle n|v\rangle = \frac{v}{2}$.

Therefore, by finding the first and second moments for $(l|x,v,n)$, we can use eq. \eqref{eq:div-varlgxv} to find the variance for $l$ given $x$ and $v$. We will use the method of the moment generating function on $P(l|x,v,n)$. The master equation for this PDF is given by

\begin{equation}
  \partial_tP(l|x,v,n) = n(x-(l-1))P(l-1|x,v,n) - (v-n)lP(l|x,v,n).
\end{equation}

Writing the moment generating function as defined on eq. \eqref{def:mom_gen} we get

\begin{equation}
  \label{eq:div-Gdef}
  G(z) \coloneqq \sum_{l=0}^xz^lP(l|x,v,n).
\end{equation}

The master eq. in terms of $G$ is thus given by \footnote{A detailed derivation of this step is done on section \ref{sec:mas-single_gene} for a more complex master equation.}

\begin{equation*}
  \partial_tG(z) = nxzG(z) - (v-n+nz)z\partial_zG(z)
\end{equation*}

At steady state we have

\begin{equation}
  \partial_zG(z) = \frac{nxz}{(v-z+nz)z}G(z)
\end{equation}

Solving with the boundary condition $G(1) = 1$, which follows from the normalization of the PDF, we find

\begin{equation}
  G(z) = \left(1+\frac{n}{v}\left(s-1\right)\right)^x = \sum_{l=0}^x {x\choose l}\left(\frac{n}{v}\right)^l\left(1-\frac{n}{v}\right)^{x-l}z^l.
\end{equation}

Comparing with eq. \eqref{eq:div-Gdef} we get

\begin{equation*}
  P(l|x,v,n) = {x\choose l}\left(\frac{n}{v}\right)^l\left(1-\frac{n}{v}\right)^{x-l},
\end{equation*}

which is a binomial distribution. Its average and variance are given by \footnote{In this case it was easy to find the distribution in terms of $G$, but we could have found the moments by differentiating and using the properties of $G$ as well.}

\begin{equation*}
  \langle l|x,v,n\rangle = \frac{n}{v}x, \quad\quad \sigma^2(L|x,v,n) = \frac{n}{v}\left(1-\frac{n}{v}\right)x.
\end{equation*}

Taking the average of the conditional variance we get

\begin{equation*}
  \begin{split}
    \left\langle\sigma^2(l|x,v,n)\right\rangle_{(n|v)} &= \left\langle\left.\frac{n}{v}\left(1-\frac{n}{v}\right)x\right|x,v,n\right\rangle_{(n|v)} = \left\langle\left.\left(\frac{n}{v}-\frac{n^2}{v^2}\right)x\right|x,v,n\right\rangle_{(n|v)}\\
    &= \left(\frac{\langle n|v\rangle}{v}-\frac{\sigma^2(n|v) + \langle n|v\rangle^2}{v^2}\right)x
  \end{split}
\end{equation*}

Where we replaced $\langle n^2|v\rangle = \sigma^2(n|v) + \langle n|v\rangle^2$. Now since $\langle n|v\rangle = \nicefrac{v}{2}$

\begin{equation}
  \label{eq:div-varofavel}
  \left\langle\sigma^2(l|x,v,n)\right\rangle_{(n|v)} = \left(\frac{1}{2} - \frac{\sigma^2(n|v)}{v^2} + \frac{1}{4}\right)x = \frac{x}{4}\left(1-Q_v^2\right),
\end{equation}

where $Q_v^2 \coloneqq \nicefrac{4\sigma^2(n|v)}{v^2}$. On the other hand, the variance of the conditional mean is given by

\begin{equation}
  \label{eq:div-aveofvarl}
  \sigma^2\left(\langle l|x,v,n\rangle\right)_{(n|v)} = \sigma^2\left(\frac{n}{v}x\right)_{(n|v)} = \frac{x^2}{v^2}\sigma^2(n|v) = \frac{x^2}{4}Q_v^2.
\end{equation}

Replacing eqs. \eqref{eq:div-varofavel} and \eqref{eq:div-aveofvarl} on eq. \eqref{eq:div-varlgxv} we get

\begin{equation*}
  \sigma^2(l|x,v) = \frac{x}{4}(1-Q_v^2)+\frac{x^2}{4}Q_v^2,
\end{equation*}

averaging and multiplying by $\nicefrac{4}{\langle x\rangle^2}$

\begin{equation}
  \label{eq:div-Qdis}
  Q_x^2 = \frac{4}{\langle x\rangle^2}\left\langle\sigma^2(l|x,v)\right\rangle = \frac{4}{\langle x\rangle^2}\frac{1}{4}\left\langle x(1-Q_v^2) + x^2Q_v^2 \right\rangle = \frac{1}{\langle x\rangle} - \frac{\langle Q_v^2x\rangle}{\langle x\rangle^2} + \frac{\langle Q_v^2x^2\rangle}{\langle x\rangle^2}.
\end{equation}

We will use this equation to calculate the partitioning error $Q_x^2$ at different scenarios.

\subsection{Random size and random accessible volume}

Consider an available volume for the molecules that varies randomly. Let $n$ be the fraction of available volume in one of the daughter cells and assume that each molecule is equally likely to occupy any volume unit. Hence, the probability per unit time of each molecule leaving its cell half is proportional to the available volume in the other cell half. Therefore the process is

\begin{equation}
  \begin{split}
    l&\xrightarrow{n(x-l)}l+1\\
    l&\xrightarrow{(1-n)l}l-1
  \end{split}
\end{equation}

This is the general case with $v=1$. Assuming the volume variance is independent of $x$, eq. \eqref{eq:div-Qdis} becomes

\begin{equation}
  Q_x^2 = \frac{1}{\langle x\rangle} - \frac{Q_v^2}{\langle x\rangle^2}\left(\langle x\rangle  - \langle x^2\rangle\right) = \frac{1}{\langle x\rangle}\left(1-\langle Q_v^2\rangle\right) + \langle Q_v^2\rangle\frac{\langle x^2\rangle}{\langle x\rangle^2},
\end{equation}

but $\frac{\langle x^2\rangle}{\langle x\rangle^2} = \frac{(\sigma^2(x) + \langle x\rangle^2)}{\langle x\rangle^2} = \eta_x^2 + 1$. Also, recalling the definition  $Q_v^2 \coloneqq \frac{4\sigma^2(n|v)}{v^2}$. In this case $v=1$ and it is fixed. Denoting it as $Q^2_\text{vol}$ we have = $Q^2_\text{vol} = Q_v^2 = \langle Q_v^2\rangle = \frac{\sigma^2(n)}{\langle n\rangle^2}$, therefore

\begin{equation}
  \boxed{Q_x^2 = \frac{1-Q_\text{vol}^2}{\langle x\rangle} + Q_\text{vol}^2(\eta^2_x+1)}.
\end{equation}

\subsection{Clustered segregation}

The clustering of molecules into vesicles could increase randomness in cell division. Let $x$ and $v$ be the total number of molecules and vesicles in a cell before division, respectively, and let $x_i$ be the number of molecules in vesicle $i$, then $\sum_{i=0}^vx_i=x$. There are two processes that produce randomness: the migration of the molecules between vesicles and the partition of the vesicles into each daughter cells.

In the first part, a vesicle loses a molecule with a probability proportional to its number of molecules.

\begin{equation}
  \label{eq:div-vesicle_switch}
  (x_1,\dotsc,x_i,\dotsc,x_j,\dotsc x_v) \xrightarrow{x_i} (x_1,\dotsc,x_i-1,\dotsc,x_j+1,\dotsc, x_v),\quad \text{for all } j\neq i.
\end{equation}

In the second part, let $n$ be the number of vesicles in one of the daughters, then similarly to the previous sections.

\begin{equation}
  \label{eq:div-arr_clust}
  \begin{split}
    n &\xrightarrow{v-n} n+1\\
    n &\xrightarrow{n} n-1
  \end{split}
\end{equation}

If we assume both processes are independent, they could be done in any order to calculate the analytical expressions. i.e. it is the same to first distribute the molecules in each vesicle and then distribute the vesicles into each cell than to first distribute the empty vesicles between cells and then distribute the molecules. We will follow the second approach.

Let $x_1,\dotsc,x_n$ be the number of molecules in each of the vesicles in one of the daughter cells and $x_{n+1},\dotsc,x_v$ be the number of molecules in the vesicles of the other daughter cell. As usual, let $l$ be the number of molecules in one of the cells, then $l = \sum_{i=1}^nx_i$, and $r = x-l = \sum_{i=n+1}^vx_i$. Therefore, among all the possible transitions of eq. \eqref{eq:div-vesicle_switch}, the transitions coming from $x_i$ and entering into $x_j$, for $i,j=1,\dotsc,n$, or $i,j=n+1,\dotsc,v$, both with $i\neq j$ does not change the number of molecules. The net effect on the number of molecules in one of the daughter cells is

\begin{equation}
  \begin{split}
    l&\xrightarrow{n(x_{n+1}+\dotsb+x_{v})}l+1,\\
    l&\xrightarrow{(v-n)(x_1+\dotsb+x_n)}l-1.
  \end{split}
\end{equation}

This can be simplified to

\begin{equation}
  \begin{split}
    l&\xrightarrow{n(x-l)}l+1\\
    l&\xrightarrow{(v-n)l}l-1,
  \end{split}
\end{equation}

which is the same as eq. \eqref{eq:div-arr_disg}. Hence, from eq. \eqref{eq:div-Qdis}

\begin{equation}
  \label{eq:div-clus-Qx1}
  Q_x^2 = \frac{1}{\langle x\rangle} - \frac{\langle Q_v^2x\rangle}{\langle x\rangle^2} + \frac{\langle Q_v^2x^2\rangle}{\langle x\rangle^2}.
\end{equation}

Also, a correspondence can be established between eq. \eqref{eq:div-arr_ind} and eq. \eqref{eq:div-arr_clust} by switching $l\leftrightarrow n$ and $x\leftrightarrow v$. Using this correspondence on eq. \eqref{eq:div-var_ind} we get 

\begin{equation}
  \sigma^2(n|v) = \frac{v}{4}.
\end{equation}

Recalling that $Q_v^2 \coloneqq \frac{4\sigma^2(n|v)}{v^2}$ we have in this case

\begin{equation}
  Q_v^2 = \frac{4}{v^2}\frac{v}{4} = \frac{1}{v},
\end{equation}

and replacing on eq. \eqref{eq:div-clus-Qx1} results in

\begin{equation}
  \label{eq:div-clus-Qx2}
  \boxed{Q_x^2 = \frac{1}{\langle x\rangle} + \frac{1}{\langle x\rangle^2}\left(\left\langle \frac{x^2}{v}\right\rangle-\left\langle \frac{x}{v}\right\rangle \right)}
\end{equation}

If $x$ and $v$ are independent we can write it as

\begin{equation*}
  \begin{split}
    Q_x^2 &= \frac{1}{\langle x\rangle} - \frac{1}{\langle x\rangle}\left\langle\frac{1}{v}\right\rangle + \frac{\langle x^2\rangle}{\langle x\rangle^2}\left\langle\frac{1}{v}\right\rangle = \frac{1}{\langle x\rangle}\left(1-\left\langle\frac{1}{v}\right\rangle\right)+\frac{\langle x^2\rangle}{\langle x\rangle^2}\left\langle\frac{1}{v}\right\rangle\\
  &\approx \frac{1}{\langle x\rangle} + \frac{\langle x^2\rangle}{\langle x\rangle^2}\left\langle\frac{1}{v}\right\rangle = \frac{1}{\langle x\rangle} + \left(1+\eta_x^2\right)\left\langle\frac{1}{v}\right\rangle,
  \end{split}
\end{equation*}

under the assumption that $\langle 1/v\rangle \ll 1$. This also allow us to do a Taylor expansion of $\langle 1/v\rangle$ around $\langle v\rangle$ obtaining

\begin{equation*}
  \begin{split}
    \left\langle\frac{1}{v}\right\rangle &\approx \left\langle \frac{1}{\langle v\rangle} - \frac{v-\langle v\rangle}{\langle v\rangle^2} + \frac{(v-\langle v\rangle)^2}{\langle v\rangle^3}\right\rangle\\
    &=\frac{1}{\langle v\rangle} + \frac{\left\langle(v-\langle v\rangle)^2\right\rangle}{\langle v\rangle^3} = \frac{1}{\langle v\rangle}\left(1+\eta_v^2\right).
  \end{split}
\end{equation*}

After replacing $Q_x$ becomes

\begin{equation*}
  \boxed{Q_x^2 \approx \frac{1}{\langle x\rangle} + \frac{(1+\eta_x^2)(1+\eta_v^2)}{\langle v\rangle}}
\end{equation*}

When $x\gg 1$, $Q_x^2\approx(1+\eta_v^2)/\langle v\rangle$, meaning that the segregation of clusters is the more significant factor on partitioning error.

Now assume that $\langle x|v\rangle = sv$ where $s$ is a constant representing the average number of molecules per vesicle. The term in parentheses of eq. \eqref{eq:div-clus-Qx2} becomes by definition of expected value

\begin{equation}
  \begin{split}
    \left\langle \frac{x^2}{v}\right\rangle-\left\langle \frac{x}{v}\right\rangle &= \sum_{x,v}\left( \frac{x^2}{v}- \frac{x}{v}\right)P(x,v) = \sum_v\frac{1}{v}\left[\sum_x\left(x^2- x\right)P(x|v)\right]P(v)\\
    &=\sum_v\frac{1}{v}\left(\langle\ x^2|v\rangle - \langle x|v\rangle\right)P(v)=\sum_v\frac{1}{v}\left(\sigma^2(x|v) + \langle x|v\rangle^2-\langle x|v\rangle\right)P(v)\\
    &=\sum_v\frac{1}{v}\left(\frac{sv\sigma^2(x|v)}{\langle x|v\rangle} + s^2v^2-sv\right)P(v) = s\left\langle\frac{\sigma^2(x|v)}{\langle x|v\rangle}\right\rangle + s^2\langle v\rangle - s
  \end{split}
\end{equation}

Defining $q\coloneqq\sigma^2(x|v)/\langle x|v\rangle$ and recalling that by the law of total expectation (eq. \eqref{eq:con-total_exp})  $\langle x\rangle = \left\langle\langle x|v\rangle\right\rangle = s\langle v\rangle$, we get

\begin{equation}
  \left\langle \frac{x^2}{v}\right\rangle-\left\langle \frac{x}{v}\right\rangle = s\langle x\rangle+s(q-1).
\end{equation}

 in eq. \eqref{eq:div-clus-Qx2} we get

\begin{equation}
  Q_x^2 = \frac{1}{\langle x\rangle} + \frac{1}{\langle x\rangle^2}\left(s\langle x\rangle+s(q-1)\right)
\end{equation}

And if $(1-q)$ is very small (or $0$ as in the case of a Poissonian), we can approximate it as

\begin{equation}
  Q_x^2 \approx \frac{1}{\langle x\rangle} + \frac{s}{\langle x\rangle}
\end{equation}

\subsection{Upper limit of the partitioning error}

There is an upper bound for the partitioning error corresponding to the case when one daughter cells keeps all the molecules. There is an equal probability of each daughter to keep all of them, hence

\begin{equation}
  \sigma^2(L|x) = \left\langle\left(l-\frac{x}{2}\right)^2\right\rangle = \frac{1}{2}\left(0-\frac{x}{2}\right)^2+\frac{1}{2}\left(x-\frac{x}{2}\right)^2 = \frac{x^2}{4}
\end{equation}

Therefore

\begin{equation}
  Q_x^2 = \frac{4}{\langle x\rangle^2}\left\langle\sigma^2(l|x)\right\rangle = \frac{4}{\langle x\rangle^2}\frac{\langle x^2\rangle}{4} = \eta_x^2+1.
\end{equation}

It only depends on the prior heterogeneity of the mother cells.

\section{Ordered segregation}

\subsection{Self-volume exclusion}

By analogy to eq. FILL making the correspondence FILL. If a molecule occupy a fixed volume fraction $k$ of the total cell volume, we have

UNDERSTAND
\begin{equation}
  \sigma^2(l|x) = \frac{1}{4}x(1-kx),
\end{equation}

so that

\begin{equation}
  Q_x^2 = \frac{4}{\langle x\rangle^2}\left\langle \sigma^2(l|x)\right\rangle = \frac{4}{\langle x\rangle^2}\frac{1}{4}\left(\langle x\rangle-k\langle x^2\rangle\right) = \frac{1}{\langle x\rangle} - k\frac{\langle x^2\rangle}{\langle x\rangle^2} = \frac{1}{\langle x\rangle} - k(\eta_x^2+1).
\end{equation}

COMMENTS. It can be noticed that the reduction with respect to independent segregation gets bigger when the volume fraction occupied by eac molecule is bigger. This makes sense because it makes the exclusion bigger when there are more molecules, having the net effect of reducing the uneveness.

\subsection{Binding to spindle sites}

Suppose each dividing cells has a random nmber of binding sites $v$ which are also distributed randomly between both daughter cells. Letting $x$ be the (random) total number of molecules of some type which are going to bind the sites before division. We will separate cells in which $v<x$ and $v\geq x$. Assume also that the binding is such that all possible molecules of $x$ are bound, that is, at equilibrium if $v<x$ all binding sites are occupied, and if $v>x$, all molecules are bound to sites.

Let $n$ be the number of binding sites on a cell half, and suppose that it increases with a rate dependent on the number of binding sites in the other cell half, then

\begin{equation}
  \begin{split}
    n\xrightarrow{f(v-n)}n+1\\
    n\xrightarrow{f(n)}n-1
  \end{split}
\end{equation}

where $f$ is some function. Also, the rate at which a molecule leaves a cell half is proportional to the number of molecules in its cell half and the number of free sites in the other cell half, obtaining

\begin{equation}
  \begin{split}
    l\xrightarrow{\lambda(n-l)(x-l)}l+1\\
    l\xrightarrow{\lambda(v-n-x+l)l}l-1
  \end{split}
\end{equation}

UNDERSTAND AND EXPLAIN. SOLVE. Solving the FDR we get

\begin{equation}
  \sigma^2(l|x,v) =\frac{1}{4}\left(x-\frac{x^2}{v}+Q_v^2x^2\right),\quad \text{for } v\geq x,
\end{equation}

with $Q_v^2 \coloneqq 4\sigma^2(n|v)/v^2$ as before. In the case $v<x$, the $v$ copies of the molecule that are bound segregate along with $n$, and for the remaining copies suppose they segregate independently. The result is (COMPARE?)

\begin{equation}
  \sigma^2(l|x,v) = \frac{1}{4}(x-v) + \sigma^2(n|v),\quad \text{for } v<x.
\end{equation}

OJO, ERROR AQUI (CREO QUE YA CORREGIDO)
Putting together both cases we get by definition of expectations

\begin{equation*}
  \begin{split}
    Q_x^2 &= \frac{4}{\langle x\rangle^2}\left\langle\sigma^2(l|x)\right\rangle = \frac{1}{\langle x\rangle^2}\sum_{x,v}\sigma^2(l|x)P(x,v)\\
    &=\frac{1}{\langle x\rangle^2}\left[\sum_{v\geq x}\left(x-\frac{x^2}{v}+Q_v^2x^2\right)P(x,v) + \sum_{v<x}\left((x-v)+4\sigma^2(n|v)\right)P(x,v)\right]\\,
  \end{split}
\end{equation*}

notice that there is an $x$ in both sums than can be taken out as a $\langle x\rangle$, also, by replacing $4\sigma^2(n|v) = v^2Q_v^2$ and separating the sums by first summing $x$ and then over all $v$s we obtain

\begin{equation}
  \begin{split}
     Q_x^2 &= \frac{1}{\langle x\rangle} - \frac{1}{\langle x\rangle^2}\sum_{v=0}^\infty\left[\sum_{x=0}^v\left(\frac{1}{v}-Q_v^2\right)x^2P(x,v)+\sum_{x=v+1}^\infty\left(v-v^2Q_v^2\right)P(x,v)\right]\\
     &=\frac{1}{\langle x\rangle} - \frac{1}{\langle x\rangle^2}\sum_{v=0}^\infty\left[\left(\frac{1}{v}-Q_v^2\right)\sum_{x=0}^vx^2P(x,v)+\left(v-v^2Q_v^2\right)\sum_{x=v+1}^\infty P(x,v)\right].
  \end{split}
\end{equation}

To make interpretations easier, consider a special case in which $v$ is fixed, each daughter cell recieves exactly $v/2$ binding sites, $\langle x\rangle = v$, and $P(x)$ is symmetric. With these assumptions, the previous eq. can be reduced

\begin{equation*}
  Q_x^2 = \frac{1}{\langle x\rangle} - \frac{1}{\langle x\rangle^2}\left[\frac{1}{v}\sum_{x=0}^vx^2P(x)+v\sum_{x=v+1}^{2v}P(x)\right]
\end{equation*}

where we made $Q_v^2=0$ because there is no uncertainty on $n$ since each cell recieves exactly $v/2$ sites. Also, for the sum over $v$ only survives the term corresponding to the fixed number $v$ of binding sites. Writing $x^2 = \left(x-\langle x\rangle\right)^2 + 2x\langle x\rangle - \langle x\rangle^2$ on the first sum we get 

\begin{equation*}
  Q_x^2 = \frac{1}{\langle x\rangle}-\frac{1}{\langle x\rangle^2}\left[\frac{\sigma^2(x)}{v}\sum_{x=0}^vP(x)+\frac{2\langle x\rangle}{v}\sum_{x=0}^vxP(x)-\frac{\langle x\rangle^2}{v}\sum_{x=0}^vP(x)+v\sum_{x=v+1}^{2v}P(x)\right]
\end{equation*}

evaluating $\langle x\rangle = v$ we get

\begin{equation*}
  Q_x^2 = \frac{1}{\langle x\rangle}-\frac{1}{\langle x\rangle^2}\left[\frac{\sigma^2(x)}{v}\sum_{x=0}^vP(x)+2\sum_{x=0}^vxP(x)-v\sum_{x=0}^vP(x)+v\sum_{x=v+1}^{2v}P(x)\right],
\end{equation*}

and since $P(x)$ is symmetric about $x=v$ we have $\sum_{x=0}^vP(x) = \sum_{x=v+1}^{2v}P(x) = 1/2$

\begin{equation}
  Q_x^2 = \frac{1}{\langle x\rangle}-\frac{1}{\langle x\rangle^2}\left[\frac{\sigma^2(x)}{2v}+2\sum_{x=0}^vxP(x)\right].
\end{equation}

The absolute deviation $\left\langle\left|x-\langle x\rangle\right|\right\rangle$ can be written using the symmetry of $P(x)$ as

\begin{equation}
  \left\langle\left|x-\langle x\rangle\right|\right\rangle = \sum_{x=0}^\infty\left|x-\langle x\rangle\right|P(x) = 2\sum_{x=0}^v\left(\langle x\rangle-x\right)P(x) = \langle x\rangle-2\sum_{x=0}^vxP(x).
\end{equation}

Thus, solving for the sum and replacing we get

\begin{equation*}
  Q_x^2 = \frac{1}{\langle x\rangle}-\frac{1}{\langle x\rangle^2}\left[\frac{\sigma^2(x)}{2v}+\langle x\rangle-\left\langle\left|x-\langle x\rangle\right|\right\rangle\right] = \frac{\left\langle\left|x-\langle x\rangle\right|\right\rangle}{\langle x\rangle^2}-\frac{\eta_x^2}{2v}
\end{equation*}

And using $v=\langle x\rangle$

\begin{equation}
  \boxed{Q_x^2 = \frac{1}{\langle x\rangle}\left(\frac{\left\langle\left|x-\langle x\rangle\right|\right\rangle}{\langle x\rangle}-\frac{\eta_x^2}{2}\right)}
\end{equation}

ANALYSIS, $\eta_x$ cannot exceed 1 by the symmetry of $P(x)$?

\subsection{Pair formation mechanisms}
A mechanism of ordered segregation consists in the pair formation of the molecules to be segregated and then spindles are formed which separates each molecule forming the pair into each cell half.

Assume that in each cell there are $z$ pairs of molecules and $m$ independent molecules i.e. $x=m+2z$. Suppose also that the paired molecules do not interact with the unpaired ones, then

\begin{equation}
  \label{eq:div-pu}
  \begin{split}
  \left\langle\sigma^2(l|x)\right\rangle &= \left\langle\sigma^2_\text{p}(L|2z)\right\rangle + \left\langle\sigma^2_\text{u}(l|m)\right\rangle = \sum_{x,z}\left[\sigma^2_\text{p}(l|2z) + \sigma^2_\text{u}(l|m)\right]P(2z,x)\\
  &= \sum_{x,z}\left[\sigma^2_\text{p}(l|2z) + \sigma^2_\text{u}(l|x-2z)\right]P(2z|x)P(x)
  \end{split}
\end{equation}

where the subscripts 'p' and 'u' stand for 'paired' and 'unpaired' and $P(2z,x)$ is the PMF of having $z$ pairs and a total of $x$ molecules before division. Now We will proceed to find each one of the variances. The unpaired molecules segregate independently, therefore, by comparison with eq. \eqref{eq:div-var_ind} we get

\begin{equation}
  \label{eq:div-u}
  \sigma^2_\text{u}(l|x-2z) = \frac{x-2z}{4}.
\end{equation}

For the paired molecules, assume that each pair is split to separate daughters with probability $p$ and to the same daughter with probability $1-p$. If the second case happends, there is an equal probability of ending in each daughter. Let $M$ be the number of unsorted molecules, and $L$ and $R$ the number of sorted molecules to each cell half. The sorting process of the pairs can be modeled with the following process

\begin{equation}
  \begin{split}
    (M,l,r)&\xrightarrow{pM}(M-2,l+1,r+1)\\
    (M,l,r)&\xrightarrow{(1-p)M/2}(M-2,l+2,r)\\
    (M,l,r)&\xrightarrow{(1-p)M/2}(M-2,l,r+2)
  \end{split}
\end{equation}

where the first line represents a succesfull split and the other two unsuccesful ones. With this relations, the matrices $\mathbf{A}$ and $\mathbf{B}$ of the time dependent FDT can be found and solved for the variances. To represent a sorting of $2z$ molecules, the initial conditions are $M(0) = 2z$, $L(0) = R(0)=0$, and all the entries of the covariance matrix are zero. The variance of the paired molecules after the sorting process is complete corresponds to the limit where the time of the mock process goes to infinity. After some algebraic steps the result is

\begin{equation}
  \label{eq:div-p}
  \sigma^2_\text{p}=(1-p)z.
\end{equation}

Replacing eqs. \eqref{eq:div-u} and \eqref{eq:div-p} in eq. \eqref{eq:div-pu} we get

\begin{equation}
  \begin{split}
    \left\langle\sigma^2(l|x)\right\rangle &=\sum_{x}\left[\sum_z\left((1-p)z+\frac{x-2z}{4}\right)P(2z|x)\right]P(x)\\
&=\sum_x\left(\frac{1-p}{2}\langle 2z|x\rangle+\frac{x-\langle 2z|x\rangle}{4}\right)P(x)\\
    &=\frac{1}{4}\sum_x\left(2(1-p)\langle 2z|x\rangle+x-\langle 2z|x\rangle\right)P(x) = \frac{1}{4}\sum_x\left(x-(2p-1)\langle 2z|x\rangle\right)P(x)\\
    &=\frac{1}{4}\left(\langle x\rangle - (2p-1)\langle 2z\rangle\right).
  \end{split}
\end{equation}

Hence,

\begin{equation}
  \boxed{Q_x^2 = \frac{1 - (2p-1)k}{\langle x\rangle}}
\end{equation}

where $k\coloneqq\langle 2z\rangle/\langle x\rangle$ is the average fraction of molecules that are in pairs. If $k=0$ there is independent segregation and $Q_x = 1/\langle x\rangle$ on the previous equation. For the segregation into pairs to be ordered, $p$ must be greater than $1/2$, in the opposite case, the paired molecules have a higher chance of not being split, increasing segregation error with respect to the independent case.

\section{Final remarks}
